\documentclass[12pt,a4paper]{article}
\usepackage[utf8]{inputenc}
\usepackage[russian]{babel}
\usepackage{amsmath}
\usepackage{amsfonts}
\usepackage{amssymb}
\usepackage{graphicx}
\usepackage{geometry}
\usepackage{listings}
\usepackage{xcolor}
\usepackage{hyperref}
\usepackage{array}
\usepackage{longtable}

\geometry{left=2cm,right=2cm,top=2cm,bottom=2cm}

\title{Практическая работа: Векторы и аналитическая геометрия \\ в прикладных задачах \\ (Задача принадлежности точки тетраэдру)}
\author{}
\date{}

\lstset{
    language=Python,
    basicstyle=\ttfamily\small,
    keywordstyle=\color{blue},
    commentstyle=\color{green!60!black},
    stringstyle=\color{red},
    numbers=left,
    numberstyle=\tiny\color{gray},
    stepnumber=1,
    numbersep=5pt,
    backgroundcolor=\color{white},
    showspaces=false,
    showstringspaces=false,
    showtabs=false,
    frame=single,
    tabsize=4,
    captionpos=b,
    breaklines=true,
    breakatwhitespace=true,
    escapeinside={\%*}{*)},
    morekeywords={def,class,return,if,elif,else,for,while,import,from,as,np}
}

\begin{document}

\maketitle

\tableofcontents
\newpage

\section{Цель}
Сравнить различные методы решения задачи принадлежности точки к тетраэдру, определить наиболее удобный и универсальный из них.

\section{Постановка задачи}
Определить, принадлежит ли заданная точка $P(x, y, z)$ тетраэдру $ABCD$. 
Исследовать случаи: внутри; на грани; на ребре; совпадение с вершиной; снаружи.

\section{Вершины пирамиды}

\subsection{Таблица координат вершин}
\begin{table}[h]
\centering
\begin{tabular}{|c|c|}
\hline
Вершина & Координаты \\
\hline
A & $(0;0;0)$ \\
B & $(1;0;0)$ \\
C & $(0;1;0)$ \\
D & $(0;0;1)$ \\
\hline
\end{tabular}
\end{table}

\subsection{Проверка корректности}
Найдем объем $V$ тетраэдра $ABCD$ через смешанное произведение векторов: $\vec{AB}, \vec{AC}, \vec{AD}$.

\[
\vec{AB} = (1,0,0), \quad \vec{AC} = (0,1,0), \quad \vec{AD} = (0,0,1)
\]

\[
\vec{AB} \cdot (\vec{AC} \times \vec{AD}) = 
\begin{vmatrix}
1 & 0 & 0 \\
0 & 1 & 0 \\
0 & 0 & 1
\end{vmatrix} = 1
\]

\[
V_{ABCD} = \frac{1}{6} \cdot 1 = \frac{1}{6} \neq 0
\]

Следовательно, тетраэдр невырожденный.

\section{Методы решения}

\subsection{Метод 1: Метод объёмов (через смешанное произведение)}

\subsubsection{Основная идея}
Вычисляем объём пирамиды $ABCD$ и объёмы пирамид с вершиной в $P$ ($PBCD$, $APCD$, $ABPD$, $ABCP$) с помощью смешанного произведения векторов. Сравниваем сумму объёмов пирамид $PBCD$, $APCD$, $ABPD$, $ABCP$ и объём пирамиды $ABCD$.

\subsubsection{Математическая реализация}
Объём пирамиды $ABCD$:
\[
V_{ABCD} = \frac{|\vec{AB} \cdot (\vec{AC} \times \vec{AD})|}{6}
\]

Объёмы пирамид с вершиной $P$:
\begin{align*}
V_1 &= V_{PBCD} = \frac{|\vec{PB} \cdot (\vec{PC} \times \vec{PD})|}{6} \\
V_2 &= V_{APCD} = \frac{|\vec{AP} \cdot (\vec{AC} \times \vec{AD})|}{6} \\
V_3 &= V_{APBD} = \frac{|\vec{AP} \cdot (\vec{AB} \times \vec{AD})|}{6} \\
V_4 &= V_{APBC} = \frac{|\vec{AP} \cdot (\vec{AB} \times \vec{AC})|}{6}
\end{align*}

Пусть $S = V_1 + V_2 + V_3 + V_4$.

\begin{itemize}
\item Если $S = V$, то точка находится внутри тетраэдра
\item Если $S = V$ и один $V_i = 0$, то точка лежит на грани
\item Если $S = V$ и два $V_i = 0$, то точка лежит на ребре
\item Если $S = V$ и три $V_i = 0$, то точка лежит на вершине
\item Если $S > V$, то точка лежит снаружи
\end{itemize}

\subsubsection{Решение вручную}
Точка $P = (\frac{1}{4}, \frac{1}{4}, \frac{1}{4})$ внутри тетраэдра.

\[
V_{ABCD} = \frac{1}{6}
\]

\[
\vec{PB} = (\frac{3}{4}, -\frac{1}{4}, -\frac{1}{4}), \quad 
\vec{PC} = (-\frac{1}{4}, \frac{3}{4}, -\frac{1}{4}), \quad 
\vec{PD} = (-\frac{1}{4}, -\frac{1}{4}, \frac{3}{4})
\]

\[
\vec{PB} \cdot (\vec{PC} \times \vec{PD}) = 
\begin{vmatrix}
\frac{3}{4} & -\frac{1}{4} & -\frac{1}{4} \\
-\frac{1}{4} & \frac{3}{4} & -\frac{1}{4} \\
-\frac{1}{4} & -\frac{1}{4} & \frac{3}{4}
\end{vmatrix} = \frac{1}{4}
\]

\[
V_{PBCD} = \frac{|1/4|}{6} = \frac{1}{24}
\]

Аналогично:
\[
V_{APCD} = \frac{1}{24}, \quad V_{APBD} = \frac{1}{24}, \quad V_{APBC} = \frac{1}{24}
\]

\[
S = \frac{1}{24} + \frac{1}{24} + \frac{1}{24} + \frac{1}{24} = \frac{1}{6} = V_{ABCD}
\]

Ни один из объемов не равен 0, значит точка $P$ лежит внутри тетраэдра.

\subsubsection{Решение с помощью кода}
\begin{lstlisting}
import math

def getVector(v1, v2):
    return (v1[0]-v2[0], v1[1]-v2[1], v1[2]-v2[2])

def vector_product(v1, v2):
    return (v1[1]*v2[2]-v1[2]*v2[1],
            v1[2]*v2[0]-v1[0]*v2[2],
            v1[0]*v2[1]-v1[1]*v2[0])

def scalar_product(v1, v2):
    return v1[0]*v2[0] + v1[1]*v2[1] + v1[2]*v2[2]

def getVolume(w,x,y,z):
    wx = getVector(x, w)
    wy = getVector(y, w)
    wz = getVector(z, w)
    det = scalar_product(wx, vector_product(wy, wz))
    return abs(det) / 6.0

def classify_point_in_tetra(A,B,C,D,P, error_margin=10**(-9)):
    V = getVolume(A,B,C,D)
    if V <= error_margin:
        return "Тетраэдр некорректный"
    V1 = getVolume(P,B,C,D)  
    V2 = getVolume(A,P,C,D)
    V3 = getVolume(A,B,P,D)
    V4 = getVolume(A,B,C,P)
    S = V1 + V2 + V3 + V4
    
    zeros = sum(1 for v in (V1,V2,V3,V4) if v <= error_margin)
    outside = abs(S - V) > error_margin
    
    if outside:
        return "Точка снаружи"
    
    if zeros == 0:
        return "Точка внутри"
    elif zeros == 1:
        return "Точка на грани"
    elif zeros == 2:
        return "Точка на ребре"
    elif zeros == 3:
        return "Точка на вершине"
    else:
        return "Вырожденный случай"

A=(0,0,0); B=(1,0,0); C=(0,1,0); D=(0,0,1)
P=(0.25,0.25,0.25)
print(classify_point_in_tetra(A,B,C,D,P))
\end{lstlisting}

\subsubsection{Выводы}
Метод объёмов позволяет определить положение точки относительно тетраэдра для всех случаев. Метод хорошо реализуется как вручную, так и программно.

\subsection{Метод 2: Метод знаков смешанных произведений}

\subsubsection{Основная идея}
Для каждой грани сравниваем знаки смешанных произведений для точки $P$ и противоположной вершины.

\subsubsection{Математическая реализация}
Для каждой грани:
\begin{itemize}
\item Грань $ABC$ (противоположная $D$): сравниваем знаки $M_{ABCP}$ и $M_{ABCD}$
\item Грань $ABD$ (противоположная $C$): сравниваем знаки $M_{ABDP}$ и $M_{ABDC}$
\item Грань $ACD$ (противоположная $B$): сравниваем знаки $M_{ACDP}$ и $M_{ACDB}$
\item Грань $BCD$ (противоположная $A$): сравниваем знаки $M_{BCDP}$ и $M_{BCDA}$
\end{itemize}

Точка $P$ принадлежит тетраэдру если для всех граней смешанные произведения имеют одинаковый знак или равны нулю.

Классификация:
\begin{itemize}
\item Внутри: все $M_P$ и $M_{opp}$ имеют одинаковый ненулевой знак
\item Снаружи: хотя бы для одной грани знаки разные
\item На грани: ровно одно $M_P = 0$
\item На ребре: два $M_P = 0$
\item На вершине: три $M_P = 0$
\end{itemize}

\subsubsection{Решение вручную}
Точка $P = (1,0,0)$ (вершина $B$).

\begin{align*}
M_{ABCP} &= \det\begin{bmatrix}1&0&0\\0&1&0\\1&0&0\end{bmatrix} = 0 \\
M_{ABCD} &= \det\begin{bmatrix}1&0&0\\0&1&0\\0&0&1\end{bmatrix} = 1 \\
M_{ABDP} &= \det\begin{bmatrix}1&0&0\\0&0&1\\1&0&0\end{bmatrix} = 0 \\
M_{ABDC} &= \det\begin{bmatrix}1&0&0\\0&0&1\\0&1&0\end{bmatrix} = -1 \\
M_{ACDP} &= \det\begin{bmatrix}0&1&0\\0&0&1\\1&0&0\end{bmatrix} = 1 \\
M_{ACDB} &= \det\begin{bmatrix}0&1&0\\0&0&1\\1&0&0\end{bmatrix} = 1 \\
M_{BCDP} &= \det\begin{bmatrix}1&-1&0\\0&-1&1\\1&-1&0\end{bmatrix} = 0 \\
M_{BCDA} &= \det\begin{bmatrix}1&-1&0\\0&-1&1\\0&-1&0\end{bmatrix} = 1
\end{align*}

Нулевых значений: 3. Точка на вершине.

\subsubsection{Решение с помощью кода}
\begin{lstlisting}
import numpy as np

def point_in_tetrahedron(P, A, B, C, D):
    vAB = B - A
    vAC = C - A
    vAP = P - A
    vAD = D - A
    M_P_ABC = np.linalg.det([vAB, vAC, vAP])
    M_D_ABC = np.linalg.det([vAB, vAC, vAD])
    M_P_ABD = np.linalg.det([vAB, vAD, vAP])
    M_C_ABD = np.linalg.det([vAB, vAD, vAC])
    M_P_ACD = np.linalg.det([vAC, vAD, vAP])
    M_B_ACD = np.linalg.det([vAC, vAD, vAB])
    vCB = B - C
    vCD = D - C
    vCP = P - C
    vCA = A - C
    M_P_BCD = np.linalg.det([vCB, vCD, vCP])
    M_A_BCD = np.linalg.det([vCB, vCD, vCA])
    epsilon = 1e-9

    def check_sign(Mp, Mopp):
        if Mopp == 0:
            return False, "Вырожденный тетраэдр"
        if Mp * Mopp < -epsilon:
            return False, "Снаружи"
        return True, "Условие выполнено"
    
    results = [
        check_sign(M_P_ABC, M_D_ABC),
        check_sign(M_P_ABD, M_C_ABD),
        check_sign(M_P_ACD, M_B_ACD),
        check_sign(M_P_BCD, M_A_BCD)
    ]

    if not all(res[0] for res in results):
        return "Снаружи"

    zero_counts = 0
    if abs(M_P_ABC) < epsilon: zero_counts += 1
    if abs(M_P_ABD) < epsilon: zero_counts += 1
    if abs(M_P_ACD) < epsilon: zero_counts += 1
    if abs(M_P_BCD) < epsilon: zero_counts += 1

    if zero_counts == 0: return "Внутри"
    elif zero_counts == 1: return "На грани"
    elif zero_counts == 2: return "На ребре"
    elif zero_counts >= 3: return "Совпадение с вершиной" 

A = np.array([0, 0, 0])
B = np.array([1, 0, 0])
C = np.array([0, 1, 0])
D = np.array([0, 0, 1])

P_inside = np.array([1/4,1/4,1/4])
P_on_face = np.array([1/3, 1/3, 1/3])
P_on_edge = np.array([0.5, 0, 0])
P_on_vertex = np.array([0, 0, 0])
P_outside = np.array([1, 1, 1])

print(f"P_inside: {point_in_tetrahedron(P_inside, A, B, C, D)}")
print(f"P_on_face: {point_in_tetrahedron(P_on_face, A, B, C, D)}")
print(f"P_on_edge: {point_in_tetrahedron(P_on_edge, A, B, C, D)}")
print(f"P_on_vertex: {point_in_tetrahedron(P_on_vertex, A, B, C, D)}")
print(f"P_outside: {point_in_tetrahedron(P_outside, A, B, C, D)}")
\end{lstlisting}

\subsubsection{Выводы}
Метод знаков смешанных произведений — надёжный инструмент анализа положения точки относительно тетраэдра. Обеспечивает точное разделение всех возможных случаев.

\subsection{Метод 3: Метод уравнений плоскостей}

\subsubsection{Основная идея}
Каждая грань тетраэдра задаётся своей плоскостью. Если точка $P$ находится по ту же сторону каждой плоскости, что и противоположная вершина, то она внутри тетраэдра.

\subsubsection{Математическая реализация}
Для каждой грани находим уравнение плоскости $Ax + By + Cz + D = 0$.

Для грани $ABC$ (противоположная вершина $D$):
\[
\vec{n} = \vec{AB} \times \vec{AC}
\]
\[
A(x - x_A) + B(y - y_A) + C(z - z_A) = 0
\]
\[
D = - (Ax_A + By_A + Cz_A)
\]

Вычисляем значение для точки $P$:
\[
F(P) = Ax_P + By_P + Cz_P + D
\]

Точка $P$ внутри тетраэдра если для всех граней:
\[
\text{sign}(F(P)) = \text{sign}(F(\text{противоположная вершина}))
\]
или $F(P) = 0$ (точка на плоскости).

\subsubsection{Решение вручную}
Точка $P = (0.25, 0.25, 0.25)$.

1. Грань $ABC$ (противоположная $D$):
\[
\vec{AB} = (1,0,0), \quad \vec{AC} = (0,1,0)
\]
\[
\vec{n} = (0,0,1), \quad D = 0
\]
Уравнение: $z = 0$
\[
F(P) = 0.25 > 0, \quad F(D) = 1 > 0
\]
Знаки совпадают.

2. Грань $ABD$ (противоположная $C$):
\[
\vec{AB} = (1,0,0), \quad \vec{AD} = (0,0,1)
\]
\[
\vec{n} = (0,-1,0), \quad D = 0
\]
Уравнение: $-y = 0$ или $y = 0$
\[
F(P) = 0.25 > 0, \quad F(C) = 1 > 0
\]
Знаки совпадают.

3. Грань $ACD$ (противоположная $B$):
\[
\vec{AC} = (0,1,0), \quad \vec{AD} = (0,0,1)
\]
\[
\vec{n} = (1,0,0), \quad D = 0
\]
Уравнение: $x = 0$
\[
F(P) = 0.25 > 0, \quad F(B) = 1 > 0
\]
Знаки совпадают.

4. Грань $BCD$ (противоположная $A$):
\[
\vec{CB} = (1,-1,0), \quad \vec{CD} = (0,-1,1)
\]
\[
\vec{n} = (-1,-1,-1), \quad D = 1
\]
Уравнение: $-x - y - z + 1 = 0$ или $x + y + z - 1 = 0$
\[
F(P) = 0.25+0.25+0.25-1 = -0.25 < 0
\]
\[
F(A) = 0+0+0-1 = -1 < 0
\]
Знаки совпадают.

Все условия выполняются, точка внутри.

\subsubsection{Решение с помощью кода}
\begin{lstlisting}
import numpy as np

def plane_equation(v1, v2, v3):
    u = np.array(v2) - np.array(v1)
    w = np.array(v3) - np.array(v1)
    n = np.cross(u, w)
    A, B, C = n
    D = -np.dot(n, v1)
    return A, B, C, D

def check_point_in_tetrahedron(P, A, B, C, D, epsilon=1e-9):
    planes = [
        (plane_equation(A, B, C), D),  # Грань ABC
        (plane_equation(A, B, D), C),  # Грань ABD
        (plane_equation(A, C, D), B),  # Грань ACD
        (plane_equation(B, C, D), A)   # Грань BCD
    ]
    
    zero_count = 0
    for (A_coef, B_coef, C_coef, D_coef), opposite_vertex in planes:
        F_P = A_coef*P[0] + B_coef*P[1] + C_coef*P[2] + D_coef
        F_opp = A_coef*opposite_vertex[0] + B_coef*opposite_vertex[1] + C_coef*opposite_vertex[2] + D_coef
        
        if abs(F_P) < epsilon:
            zero_count += 1
            continue
            
        if F_P * F_opp < -epsilon:
            return "Снаружи"
    
    if zero_count == 0:
        return "Внутри"
    elif zero_count == 1:
        return "На грани"
    elif zero_count == 2:
        return "На ребре"
    elif zero_count == 3:
        return "На вершине"
    else:
        return "Вырожденный случай"

A = (0, 0, 0)
B = (1, 0, 0)
C = (0, 1, 0)
D = (0, 0, 1)

test_points = [
    ((0.25, 0.25, 0.25), "Внутри"),
    ((1/3, 1/3, 1/3), "На грани BCD"),
    ((0.5, 0, 0), "На ребре AB"),
    ((0, 0, 0), "На вершине A"),
    ((1, 1, 1), "Снаружи")
]

for point, expected in test_points:
    result = check_point_in_tetrahedron(point, A, B, C, D)
    print(f"Точка {point}: {result} (ожидалось: {expected})")
\end{lstlisting}

\subsubsection{Выводы}
Метод уравнений плоскостей интуитивно понятен и легко реализуем. Позволяет определить положение точки относительно тетраэдра через проверку знаков значений в уравнениях плоскостей.

\subsection{Метод 4: Метод барицентрических координат}

\subsubsection{Основная идея}
Любую точку внутри тетраэдра можно представить как взвешенную сумму вершин с коэффициентами (барицентрическими координатами) $\alpha, \beta, \gamma, \delta$, такими что:
\[
P = \alpha A + \beta B + \gamma C + \delta D
\]
где $\alpha + \beta + \gamma + \delta = 1$.

Точка $P$ принадлежит тетраэдру тогда и только тогда, когда все барицентрические координаты неотрицательны.

\subsubsection{Математическая реализация}
Выразим барицентрические координаты через объёмы тетраэдров:
\[
\alpha = \frac{V_{PBCD}}{V_{ABCD}}, \quad 
\beta = \frac{V_{APCD}}{V_{ABCD}}, \quad 
\gamma = \frac{V_{ABPD}}{V_{ABCD}}, \quad 
\delta = \frac{V_{ABCP}}{V_{ABCD}}
\]

Где $V_{ABCD}$ — объём исходного тетраэдра.

Условия принадлежности:
\begin{itemize}
\item Все $\alpha, \beta, \gamma, \delta \geq 0$
\item $\alpha + \beta + \gamma + \delta = 1$
\end{itemize}

Классификация положения:
\begin{itemize}
\item Внутри: все координаты $> 0$
\item На грани: одна координата $= 0$, остальные $> 0$
\item На ребре: две координаты $= 0$, остальные $> 0$
\item На вершине: три координаты $= 0$, одна $= 1$
\item Снаружи: хотя бы одна координата $< 0$
\end{itemize}

\subsubsection{Решение вручную}
Точка $P = (0.25, 0.25, 0.25)$.

Из метода 1 мы уже вычислили:
\[
V_{ABCD} = \frac{1}{6}, \quad 
V_{PBCD} = \frac{1}{24}, \quad 
V_{APCD} = \frac{1}{24}, \quad 
V_{ABPD} = \frac{1}{24}, \quad 
V_{ABCP} = \frac{1}{24}
\]

Барицентрические координаты:
\[
\alpha = \frac{1/24}{1/6} = \frac{1}{4}, \quad
\beta = \frac{1/24}{1/6} = \frac{1}{4}, \quad
\gamma = \frac{1/24}{1/6} = \frac{1}{4}, \quad
\delta = \frac{1/24}{1/6} = \frac{1}{4}
\]

Проверяем условия:
\begin{itemize}
\item Все координаты $\frac{1}{4} > 0$ — выполняется
\item Сумма: $\frac{1}{4} + \frac{1}{4} + \frac{1}{4} + \frac{1}{4} = 1$ — выполняется
\end{itemize}

Точка внутри тетраэдра.

Проверим точку $P = (0.5, 0, 0)$ (на ребре $AB$):
\[
V_{PBCD} = 0 \quad (\text{точка на плоскости BCD})
\]
\[
V_{APCD} = \frac{|\vec{AP} \cdot (\vec{AC} \times \vec{AD})|}{6} = \frac{|0.5 \cdot 1|}{6} = \frac{0.5}{6} = \frac{1}{12}
\]
\[
V_{ABPD} = 0 \quad (\text{точка на плоскости ABD})
\]
\[
V_{ABCP} = \frac{|\vec{AP} \cdot (\vec{AB} \times \vec{AC})|}{6} = \frac{|0.5 \cdot 1|}{6} = \frac{0.5}{6} = \frac{1}{12}
\]

\[
\alpha = \frac{0}{1/6} = 0, \quad
\beta = \frac{1/12}{1/6} = 0.5, \quad
\gamma = \frac{0}{1/6} = 0, \quad
\delta = \frac{1/12}{1/6} = 0.5
\]

Две координаты равны 0, точка на ребре.

\subsubsection{Решение с помощью кода}
\begin{lstlisting}
import numpy as np

def barycentric_coordinates(P, A, B, C, D, epsilon=1e-9):
    def tetra_volume(v1, v2, v3, v4):
        v12 = v2 - v1
        v13 = v3 - v1
        v14 = v4 - v1
        det = np.linalg.det([v12, v13, v14])
        return abs(det) / 6.0
    
    V_total = tetra_volume(A, B, C, D)
    
    if V_total < epsilon:
        return None, "Вырожденный тетраэдр"
    
    alpha = tetra_volume(P, B, C, D) / V_total
    beta = tetra_volume(A, P, C, D) / V_total
    gamma = tetra_volume(A, B, P, D) / V_total
    delta = tetra_volume(A, B, C, P) / V_total
    
    return [alpha, beta, gamma, delta], None

def classify_by_barycentric(P, A, B, C, D, epsilon=1e-9):
    coords, error = barycentric_coordinates(P, A, B, C, D, epsilon)
    
    if error:
        return error
    
    alpha, beta, gamma, delta = coords
    
    negative_count = sum(1 for c in coords if c < -epsilon)
    zero_count = sum(1 for c in coords if abs(c) < epsilon)
    positive_count = sum(1 for c in coords if c > epsilon)
    
    if negative_count > 0:
        return "Снаружи"
    
    if zero_count == 0:
        return "Внутри"
    elif zero_count == 1:
        return "На грани"
    elif zero_count == 2:
        return "На ребре"
    elif zero_count == 3:
        return "На вершине"
    else:
        return "Вырожденный случай"

def point_location_barycentric(P, A, B, C, D, epsilon=1e-9):
    coords, error = barycentric_coordinates(P, A, B, C, D, epsilon)
    
    if error:
        return error, None
    
    alpha, beta, gamma, delta = coords
    
    location = classify_by_barycentric(P, A, B, C, D, epsilon)
    
    return location, {
        'alpha': alpha,
        'beta': beta,
        'gamma': gamma,
        'delta': delta,
        'sum': alpha + beta + gamma + delta
    }

A = np.array([0, 0, 0])
B = np.array([1, 0, 0])
C = np.array([0, 1, 0])
D = np.array([0, 0, 1])

test_points = [
    (np.array([0.25, 0.25, 0.25]), "Внутри"),
    (np.array([1/3, 1/3, 1/3]), "На грани"),
    (np.array([0.5, 0, 0]), "На ребре"),
    (np.array([0, 0, 0]), "На вершине"),
    (np.array([1, 1, 1]), "Снаружи"),
    (np.array([0.1, 0.2, 0.3]), "Внутри"),
]

print("Метод барицентрических координат:")
print("-" * 50)

for P, expected in test_points:
    location, info = point_location_barycentric(P, A, B, C, D)
    
    if info:
        print(f"Точка {P}:")
        print(f"  Положение: {location} (ожидалось: {expected})")
        print(f"  Координаты: α={info['alpha']:.3f}, β={info['beta']:.3f}, γ={info['gamma']:.3f}, δ={info['delta']:.3f}")
        print(f"  Сумма: {info['sum']:.3f}")
        print()
    else:
        print(f"Точка {P}: {location}")
        print()
\end{lstlisting}

\subsubsection{Выводы}
Метод барицентрических координат обладает следующими преимуществами:
\begin{enumerate}
\item Прямая геометрическая интерпретация через объёмы
\item Легко определить положение точки (внутри, на грани, на ребре, на вершине)
\item Коэффициенты могут быть использованы для интерполяции свойств (цвет, текстура) в компьютерной графике
\item Устойчив к вычислительным погрешностям при использовании относительных объёмов
\end{enumerate}

Недостатки:
\begin{enumerate}
\item Требует вычисления 5 объёмов (4 маленьких и 1 полного)
\item Менее эффективен чем метод уравнений плоскостей для множественных проверок
\item Чувствителен к вычислительным погрешностям при малых объёмах
\end{enumerate}

\subsection{Тестовые точки}

\begin{table}[h]
\centering
\begin{tabular}{|c|c|c|}
\hline
Координаты точки P & Расположение & Примечание \\
\hline
$(0.25, 0.25, 0.25)$ & Внутри & Центр масс \\
$(0.33, 0.33, 0.33)$ & На грани & На плоскости $x+y+z=1$ \\
$(0.5, 0, 0)$ & На ребре & На ребре $AB$ \\
$(0, 0, 0)$ & Вершина & Вершина $A$ \\
$(1, 1, 1)$ & Снаружи & За пределами тетраэдра \\
$(0, 0.5, 0.5)$ & На грани & На плоскости $x=0$ \\
$(0.5, 0.5, 0)$ & На грани & На плоскости $z=0$ \\
$(0, 0, 0.5)$ & На ребре & На ребре $AD$ \\
$(0.1, 0.2, 0.3)$ & Внутри & Произвольная внутренняя точка \\
$(0.7, 0.1, 0.1)$ & Снаружи & За гранью $x+y+z=1$ \\
\hline
\end{tabular}
\end{table}

\subsection{Сравнение методов}
\begin{table}[h]
\centering
\begin{tabular}{|p{3.5cm}|p{4.5cm}|p{4.5cm}|}
\hline
Метод & Плюсы & Минусы \\
\hline
\textbf{Метод объёмов} & 
\begin{itemize}
\vspace{-4mm}
\item Прямая геометрическая интерпретация
\item Работает для всех случаев
\item Легко считать вручную для простых точек
\end{itemize} &
\begin{itemize}
\vspace{-4mm}
\item Чувствителен к вычислительным погрешностям
\item Требует вычисления 5 определителей
\item Сложнее для невыпуклых многогранников
\end{itemize} \\
\hline
\textbf{Метод знаков смешанных произведений} &
\begin{itemize}
\vspace{-4mm}
\item Чёткая математическая основа
\item Позволяет определить положение на грани/ребре/вершине
\item Эффективная реализация
\end{itemize} &
\begin{itemize}
\vspace{-4mm}
\item Требует аккуратной работы с знаками
\item Сложнее для понимания начинающими
\item Чувствителен к ориентации тетраэдра
\end{itemize} \\
\hline
\textbf{Метод уравнений плоскостей} &
\begin{itemize}
\vspace{-4mm}
\item Интуитивно понятный
\item Легко расширяется на произвольные многогранники
\item Хорошо работает с отсечениями в графике
\end{itemize} &
\begin{itemize}
\vspace{-4mm}
\item Требует предварительного вычисления уравнений плоскостей
\item Нужно правильно определять направление нормалей
\item Больше вычислений для проверки
\end{itemize} \\
\hline
\textbf{Метод барицентрических координат} &
\begin{itemize}
\vspace{-4mm}
\item Естественная геометрическая интерпретация
\item Коэффициенты могут использоваться для интерполяции
\item Позволяет легко определить положение точки
\end{itemize} &
\begin{itemize}
\vspace{-4mm}
\item Требует вычисления 5 объёмов
\item Менее эффективен для множественных проверок
\item Чувствителен к погрешностям при малых объёмах
\end{itemize} \\
\hline
\end{tabular}
\caption{Сравнение методов проверки принадлежности точки тетраэдру}
\label{tab:methods_comparison}
\end{table}
\subsection{Вывод}
Наиболее удобным и универсальным является \textbf{Метод уравнений плоскостей}, так как он:
\begin{enumerate}
\item Легко обобщается на произвольные выпуклые многогранники
\item Интуитивно понятен геометрически
\item Широко применяется в компьютерной графике для отсечения
\item Позволяет легко определить расстояние до граней при необходимости
\end{enumerate}

\textbf{Метод барицентрических координат} особенно полезен в компьютерной графике для интерполяции цветов, текстур и других свойств вершин.

Для простых задач проверки принадлежности точки тетраэдру достаточно \textbf{Метода объёмов}, но для работы с более сложными полиэдрами и в прикладных задачах компьютерной графики предпочтительнее метод уравнений плоскостей или барицентрических координат.

\section{Формулировка прикладной задачи}

\subsection{Область: Компьютерная графика}

В компьютерной графике при работе с виртуальной камерой часто используется понятие \textbf{пирамиды видимости} (view frustum). Это усечённая пирамида, определяющая область пространства, которая видна через камеру. В упрощённом случае для точечного источника света или для некоторых алгоритмов определения видимости используется полная пирамида (тетраэдр).

\textbf{Задача:} Определить, находится ли объект (или его часть) внутри пирамиды видимости камеры. Это необходимо для:
\begin{enumerate}
\item \textbf{Отсечения невидимых объектов:} Если объект полностью вне пирамиды видимости, его не нужно отрисовывать
\item \textbf{Определения уровня детализации:} Объекты ближе к камере требуют более детальной прорисовки
\item \textbf{Оптимизации физических вычислений:} Взаимодействие рассчитывается только для объектов в поле зрения
\end{enumerate}

\textbf{Решение:} Для каждого объекта вычисляется его ограничивающий параллелепипед (bounding box) или сфера. Проверяем принадлежность вершин этого объёма к пирамиде видимости (тетраэдру). Если хотя бы одна вершина внутри — объект потенциально видим.

\subsection{Область: Робототехника}

В робототехнике при планировании траектории манипулятора важно обеспечить безопасность работы. Определяются \textbf{запретные зоны} — области пространства, куда не должен заходить манипулятор.

\textbf{Задача:} Проверить, не заходит ли рабочий орган робота в запретную зону, заданную в виде тетраэдра (например, зона вокруг опасного оборудования или хрупких объектов).

\textbf{Решение:} В реальном времени отслеживается положение конечного эффектора манипулятора. Проверяется принадлежность этой точки к запретным зонам. При обнаружении пересечения система безопасности останавливает робота или корректирует траекторию.

\section{Визуализация результатов}

Для визуализации использован \textbf{Python} с библиотеками \textbf{matplotlib} и \textbf{numpy}.

\begin{lstlisting}
import numpy as np
import matplotlib.pyplot as plt
from mpl_toolkits.mplot3d import Axes3D

def plot_tetrahedron(ax, A, B, C, D):
    vertices = np.array([A, B, C, D])
    
    edges = [
        [0, 1], [0, 2], [0, 3],
        [1, 2], [1, 3], [2, 3]
    ]
    
    for edge in edges:
        ax.plot3D(*zip(vertices[edge[0]], vertices[edge[1]]), 'b-', linewidth=2)
    
    ax.scatter(*zip(A, B, C, D), c='red', s=100, marker='o')
    
    labels = ['A', 'B', 'C', 'D']
    for i, label in enumerate(labels):
        ax.text(vertices[i][0], vertices[i][1], vertices[i][2], label, fontsize=12)

def plot_test_points(ax, points, colors, markers):
    for point, color, marker in zip(points, colors, markers):
        ax.scatter(*point, c=color, s=150, marker=marker, alpha=0.8)

A = np.array([0, 0, 0])
B = np.array([1, 0, 0])
C = np.array([0, 1, 0])
D = np.array([0, 0, 1])

test_points = [
    np.array([0.25, 0.25, 0.25]),  # Внутри - зеленый круг
    np.array([1/3, 1/3, 1/3]),     # На грани - желтый квадрат
    np.array([0.5, 0, 0]),         # На ребре - оранжевый треугольник
    np.array([0, 0, 0]),           # Вершина - красная звезда
    np.array([1, 1, 1]),           # Снаружи - черный крест
]

colors = ['green', 'yellow', 'orange', 'red', 'black']
markers = ['o', 's', '^', '*', 'x']

fig = plt.figure(figsize=(12, 10))
ax = fig.add_subplot(111, projection='3d')

plot_tetrahedron(ax, A, B, C, D)
plot_test_points(ax, test_points, colors, markers)

ax.set_xlabel('X')
ax.set_ylabel('Y')
ax.set_zlabel('Z')
ax.set_title('Визуализация тетраэдра и тестовых точек')
ax.view_init(elev=20, azim=45)

plt.legend(['Тетраэдр', 'Внутри', 'На грани', 'На ребре', 'На вершине', 'Снаружи'],
           loc='upper right')
plt.tight_layout()
plt.show()
\end{lstlisting}

\begin{figure}[h]
\centering
\includegraphics[width=0.8\textwidth]{tetrahedron_visualization.png}
\caption{Визуализация тетраэдра и тестовых точек различных типов}
\end{figure}

\textbf{Обозначения на рисунке:}
\begin{itemize}
\item Синие линии — рёбра тетраэдра
\item Красные точки — вершины тетраэдра (A, B, C, D)
\item Зелёный круг — точка внутри тетраэдра
\item Жёлтый квадрат — точка на грани
\item Оранжевый треугольник — точка на ребре
\item Красная звезда — точка на вершине
\item Чёрный крест — точка снаружи тетраэдра
\end{itemize}

\end{document}
