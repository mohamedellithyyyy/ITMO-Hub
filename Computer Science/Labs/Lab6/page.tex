
\setcounter{page}{38}

\begin{paracol}{3}

\begin{center}
\includegraphics[width=\linewidth]{images/pic1.png}
\end{center}

\switchcolumn

\begin{center}
\includegraphics[width=\linewidth]{images/pic2.png}
\end{center}

\switchcolumn

\begin{center}
\includegraphics[width=\linewidth]{images/pic3.png}
\end{center}

\end{paracol}

\begin{paracol}{2}
При этом иногда на «естественном» 
чертеже (т. е. на чертеже, на котором 
только «изображено» условие) трудно
заметить связи между данными и ис-
комыми величинами, а если фигуру 
«достроить», эти связи становятся оче-
видными.

$\text{З а д а ч а 3}$ (МГУ, ф-т почво-
ведения, 1977). Длины оснований $CD$, 
диагонали $BD$ и боковой стороны AD
трапеции $ABCD$ равны между со-
бой и равны p. Длина боковой стороны
$BC$ равна q. Найти длину диагонали
$AC$.

В данной трапеции $ABCD$ (рис. 4)
нелегко увидеть связь между искомой 
диагональю $AC$ и другими отрезками. 
Если же, приняв во внимание, что 
точка $D$ равноудалена от точек $A$,
$B$ и $C$, провести окружность $O$$(D, p)$
и «достроить» данную трапецию до
равнобедренной трапеции $ABCE$, из
прямоугольного треугольника $ACE$ 
легко найдём $|AC| = \sqrt{4p^2 - q^2}$ .

Задача 4 (НГУ, 1976). В тра-
пеции $ABCD$ с основаниями $AB$ и
$CD$ биссектриса угла B перпендику-
лярна боковой стороне $AD$ и пере-
секает её в точке $E$. В каком отноше-

\switchcolumn
нии прямая BE делит площадь тра-
пеции, если известно, что длина от-
резка AE в два раза больше длины от-
резка $DE$?
  
Если «достроить» данную трапе-
цию $ABCD$ до треугольника $AFB$
(рис. 5), получим равнобедренный
треугольник, от которого отрезок $DC$
отсекает подобный треугольник $DFC$
с коэффициентом подобия \(\tfrac{1}{4}\). По-
этому искомое отношение равно

\begin{align*}
\frac{S_{EBCS}}{S_{\triangle AEB}} 
&= \frac{S_{\triangle EFB} - S_{\triangle DFC}}{S_{\triangle AEB}} \\[2mm]
&= \frac{\frac{1}{2} S_{\triangle AFB} - \left(\frac{1}{4}\right)^2 S_{\triangle AFB}}{S_{\triangle AEB}} =  \frac{7}{8}
\end{align*}

\\ 3. Опишем окружность.
\\ В некоторых случаях существенным моментом в геометрическом решении
задачи является установление кон-
груэнтности некоторых углов. Чаще 
всего такие углы являются соответ-
ственными в подобных треугольниках.
\end{paracol}

\vspace{1cm}

% -----------------------
% THIRD SECTION — 3 COLUMNS AGAIN
% -----------------------

\begin{paracol}{3}

\begin{center}
\includegraphics[width=\linewidth]{images/pic4.png}
\end{center}

\switchcolumn

\begin{center}
\includegraphics[width=\linewidth]{images/pic5.png}
\end{center}

\switchcolumn

\begin{center}
\includegraphics[width=\linewidth]{images/pic6.png}
\end{center}

\end{paracol}



\end{document}
